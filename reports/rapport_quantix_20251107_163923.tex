
        \documentclass[12pt,a4paper]{article}
        \usepackage[T1]{fontenc}
        \usepackage[utf8]{inputenc}
        \usepackage[french]{babel}
        \usepackage{lmodern}
        \usepackage{geometry}
        \usepackage{titlesec}
        \usepackage{xcolor}
        \usepackage{fancyhdr}
        \usepackage{setspace}
        \usepackage{hyperref}
        \usepackage{booktabs}
        \usepackage{array}

        \geometry{margin=2.5cm}
        \pagestyle{fancy}
        \fancyhead[L]{\textbf{Quantix}}
        \fancyhead[R]{Rapport d'analyse}
        \fancyfoot[C]{Page \thepage}

        \definecolor{qblue}{HTML}{005B96}
        \definecolor{qgray}{HTML}{F5F5F5}

        \titleformat*{\section}{\Large\bfseries\color{qblue}}
        \titleformat*{\subsection}{\large\bfseries\color{qblue}}
        \renewcommand{\baselinestretch}{1.25}

        \begin{document}
        \begin{center}
        {\LARGE\textbf\color{qblue}{Rapport d'Analyse Quantix}}\\[0.5em]
        {\large Généré le 07 November 2025 à 16:39}\\[1em]
        \rule{\textwidth}{1pt}
        \end{center}
        \vspace{1em}

        \section{Rapport D'Analyse Quantix}
\textit{Généré le 07/11/2025 à 16}: 39:23\\

\vspace{{0.5em}}\hrule\vspace{{0.5em}}
\section{Fichier Analyse}
Ce préambule décrit la source des données, sa matérialité et les métadonnées essentielles pour la traçabilité.\\
\textbf{Nom: N/A}\\
\textbf{Taille: 0.00 MB}\\
\textbf{Modifié: N/A}\\
Le nom, la taille et la date de modification facilitent la réconciliation des versions et la reproductibilité des traitements.\\

\section{Structure Des Donnees}
\textit{Panorama structurel du tableau}: dimensions, empreinte mémoire et nature typée des colonnes.\\
\textbf{Dimensions: 500 lignes × 5 colonnes}\\
\textbf{Mémoire utilisée: 0.15 MB}\\
\textbf{Types de données: 5 différents}\\
\textit{Exemple de colonnes et types (échantillon non exhaustif)}: \\
\textit{Nom}: object\\
\textit{Âge}: object\\
\textit{Revenu}: object\\
\textit{Date\textbackslash{}_inscription}: object\\
\textit{Commentaire}: object\\
Une forte hétérogénéité des types impose des pipelines de normalisation explicites avant toute modélisation.\\

\section{Qualite Des Donnees}
Synthèse des défauts observés et de leur gravité opérationnelle. Elle guide les arbitrages de nettoyage.\\
\textbf{Complétude: 98.80\textbackslash{}% (31 valeurs manquantes)}\\
\textbf{Doublons: 28 lignes identiques}\\
\textbf{Colonnes avec manques: 2}\\
\textit{Colonnes les plus affectées par les valeurs manquantes (top 5)}: \\
\textit{Âge}: 27 manques\\
\textit{Commentaire}: 4 manques\\
\textit{Interprétation}: la complétude mesure la part de champs renseignés. Les doublons biaiseront toute agrégation.\\
\textit{Action prioritaire}: traiter d'abord les colonnes motrices du produit (identifiants, dates, montants) puis la volumétrie de manques.\\

\section{Historique Des Traitements}
Journal d'exécution des opérations. Utile pour l'audit et la reproductibilité.\\
\textbf{Total opérations: 0}\\
\textbf{Durée totale: 0.00 secondes}\\
\textbf{Catégories utilisées: Aucune}\\
Cet impact cumulé matérialise le gain de qualité. Il doit converger vers des indicateurs stables d'une exécution à l'autre.\\

\section{Analyse Interprétative}
Cette section articule constats, risques et conséquences métier pour soutenir une décision rationnelle.\\
\textbf{Risque de biais: valeurs manquantes non aléatoires susceptibles de fausser les distributions et les modèles.}\\
\textbf{Traçabilité: l'historique des traitements garantit l'explicabilité et la conformité.}\\
\textbf{Préparation aux modèles: typage clair, granularité cohérente, dates normalisées et catégories contrôlées sont nécessaires.}\\
En pratique, viser une complétude > 98\textbackslash{}% sur les champs critiques et zéro doublon strict sur les identifiants.\\

\section{Recommandations}
Les actions ci-dessous sont ordonnées du plus structurant au plus simple, afin de maximiser le gain qualité/coût.\\
\textbf{Normaliser les schémas: dictionnaire de données et conventions de nommage stables}\\
\textbf{Nettoyer les valeurs manquantes: imputation ciblée (médiane/mode) ou suppression raisonnée}\\
\textbf{Éliminer les doublons: clé technique puis clés métier}\\
\textbf{Harmoniser les types: dates ISO 8601, numériques décimaux, catégories contrôlées}\\
\textbf{Détecter les outliers: règles robustes (IQR, z-score tronqué) puis validation métier}\\
\textbf{Mettre en place des tests de qualité récurrents: seuils, alertes, rapports programmés}\\
Chacune de ces mesures réduit l'entropie informationnelle et améliore la prédictibilité en aval.\\

Quantix - Nettoyage de données intelligent\\

        \vspace{2em}
        \begin{center}
        \rule{0.5\textwidth}{0.5pt}\\
        \textit{Rapport généré automatiquement par Quantix}
        \end{center}

        \end{document}
